\section{Client Library, Utilities and Extensions (CLUE) for Microsoft .Net}
\label{sec:app}

In the previous section we described how to type URL's into a Web
browser to issue commands to the Agent directly.  This section describes
how to write client applications for \mtc{} that will do this
programmatically, using the \mtc{} Client Library, Utilities and
Extensions (CLUE) for Microsoft .NET Framework~2.0 or
later.\footnote{Future versions of CLUE may include support for other
  development environments and languages such as Java, Python and Ruby.}

As you already know, \emph{you don't need any SDK to write MTConnect
  applications}, as long as your programming language or development
system includes libraries for using XML, XPath and HTTP.  The CLUE
libraries and the API's they provide, even though not part of the
official \mtc{} speification, may help streamline your code.  CLUE uses
.NET's underlying HTTP and XML/XPath libraries to automate and
streamline some of the procedures you learned about in the previous
section.

\boxednote{In fact, once you've made it all the way to the end of this
  tutorial, you may find it helpful to peruse the source code of CLUE to
  understand just how the CLUE API calls are converted to HTTP, XML and
  XPath calls.}

A unique feature of the Microsoft .NET framework is that libraries
compiled for it can be called from applications in any language
supported by .NET.  Currently these include C\# (C Sharp), J\# (similar
to Java), Visual Basic.NET, and many, many others.  Our examples will
use C\#, but translation to other .NET-supported languages is
straightforward.

% \boxednote{\textbf{Why aren't there ``convenience libraries'' planned
% for ANSI~C
%     programmers?}  We've chosen to
%   deemphasize ``plain'' C because string-intensive
%   and networking operations are considerably more cumbersome in C than
%   they are in Java, .NET, or modern scripting languages such as Perl or
%   Python; because the class library support for these functions is still
%   lacking in C compared to other languages; and because on modern hardware
%   it is unlikely that C's performance advantage will be a
%   dominant factor in overall performance.}

\subsection{Overview of CLUE for .NET}

We will use C\# for most
examples, occasionally showing the equivalent VB.Net code.  There are
three object abstractions provided by the CUL (and therefore three
classes provided) in the namespace MTConnect:

\begin{enumerate}
\item MTConnect.MTCClient: represents a connection to an \mtc{}
  Agent. Remembers the Agent's host and port number and the ``last sequence
  number'' sample seen from the Agent (section~\ref{sec:cmd:sample}).
  Provides methods for sending commands such as Probe, Sample, etc.  An
  application that communicates with multiple Agents would create
  multiple instances of this object.
\item MTCResponse: a class that encapsulates a successful response from
  the Agent.  Given a response message, this class provides methods that
  let you easily extract specific measurements using just the names of
  the components (e.g. ``Linear, X'' to get an axis position
  measurement), or if you wish, get the raw XML.
\item MTCEvent: given a measurement extracted from an MTCResponse,
  provides functions to perform unit conversions by cross-referencing
  the Probe schema.  Converts attribute values to native
  classes (e.g. converts the \x{timestamp} attribute to a .Net
  \icode{DateTime} object on which time calculations can be performed).
\end{enumerate}

A typical application using the .NET CLUE Libraries
has the following structure:

\begin{enumerate}
\item Create an instance of an \icode{MTCAgent} object to represent the
  connection 
  to a particular Agent.
\item Issue a Probe command,  which returns an \icode{MTCProbeResponse}
  object.  Methods on this object tell you about the device(s) connected
  to the agent, including the devices' manufacturers, serial numbers,
  and any other MTConnect-provided information.  It also provides 
  CLUE the information that will be necessary later
  to associate the correct measurement units with returned samples.
\item While measurements are still needed:
  \begin{enumerate}
  \item Optionally create an XPath expression corresponding to the desired
    measurements (or use no expression, to get all reportable measurements);
  \item Call the \icode{Sample} or \icode{Current} method on the Agent
    object---both of these return an \icode{MTCSampleResponse} object;
  \item Call \icode{FindByComponentAndName} on the
    \icode{MTCSampleResponse} object to extract desired measurements,
    which are returned as \icode{MTCEvent} objects that include the
    measurement value, units, lane, and other information.
  \end{enumerate}
\end{enumerate}

\subsection{Example 1: Hello World for \mtc}
\label{sec:ex:helloworld}

This first example repeatedly fetches measurement values for a
particular
component type and name, 
e.g. a \x{Linear} type axis named \x{Z}.
To keep it simple, in this version we will use the
Current command with no arguments (section~\ref{sec:cmd:current}).

The reason you must specify both the Component type and Name for a
desired measurement is that the \mtc{} specification guarantees that any
given \emph{combination} of Component and Name must be unique within a
device, but either Component or Name by itself might not be.  So there
might be more than one Component of type Linear (e.g., there could be X,
Y and Z linear axes), and more than one
Component whose name is \x{Z} (e.g., a machine-specific property named
Z), but there can be at most one component whose
type is Linear \emph{and} whose name is \x{Z}.  Later we will see
another way to constrain what measurement is desired, by passing a Path
argument to \icode{Current}.

Here is some example output from that program, where each new set of
measurements is delivered after each press of the Return key:

\subsubsection{Things To Notice}

CLUE tries to automate some common ``housekeeping'' tasks even in a
simple program like this example.

\paragraph{Next Sequence.} Along with each set of measurements, the
program displays the ``next sequence''.  Recall the ``bucket'' model
whereby the Agent ``fills the bucket'' with measurements and a client
application removes them.  The ``next sequence'' number is the number of
the next sample the client has not yet seen.  Later when we use the
Sample command, we'll show how to use this to make sure we don't miss
any samples.  The \icode{MTCAgent} object automatically parses this
value from the output of every Sample or Current command (e.g., see
line~3 of figure~\ref{fig:current-output}) and remembers it for you.

\paragraph{Units.} The units of each measurement, if known, are
reported.  The first time you do a Probe, the \icode{MTCAgent} object
automatically remembers the unit types associated with each measurement
(e.g., see lines~13, 18, 25, etc. in figure~\ref{fig:probe-output})



\subsubsection{Using Fewer Constraints}

Change lines 23--24 of Example~\ref{fig:code:helloworld}  as follows:

\begin{verbatim}
 23   String name = String.Empty;
 24   String comp = "Linear";

\end{verbatim}

Then rebuild and re-run the program.  Now, each time you press Return
you should see a total of 6 measurements per sample: a commanded and an
actual position for each of the X, Y and Z Linear axes.

What happened? Because we did not specify a particular component name
(we used \icode{String.Empty}) in line 23, we got samples for \emph{all}
components whose component type is \icode{Linear}.  In fact, this would
correspond directly to the XPath expression 
\icode{//ComponentStream[@component="Linear"]}, and you can browse the
source code of CLUE (file \texttt{Client.cs}) to
verify that this is exactly what's being done.

\boxednote{Why use String.Empty?}{\icode{String.empty} is the
  ``natural'' way of declaring 
  an empty string in C\#; you could also pass
  the empty string literal \texttt{""}, but \icode{String.empty} is more
  efficient since it doesn't create a new object.  Also, CLUE
  also allows you to pass the value
  \icode{null} in this case, 
  though in general \icode{null} and the empty string are
  \emph{not} interchangeable.}

If you set \icode{comp} to the empty string but pass a value (such as
\icode{"X"}) for \icode{name}, you'll get measurement samples for all
components regardless of their type whose name matches \icode{"X"}.
(For example, if the same Agent is reporting on multiple distinct sets
of linear axes, more than one of which has an axis called \icode{X},
this combination of arguments would give back samples for all of the X
axes.)


\subsubsection{Other Ways To Use \icode{FindByComponentAndName}}

An alternative version of \icode{FindByComponentAndName} allows
you to specify additional criteria to match on certain attributes.  For
example, if you know you only want to look at actual positions or
velocities and not commanded ones, i.e. position or velocity
measurements whose \emph{subType} attribute matches \icode{ACTUAL}, you
could pass the XPath expression
\icode{//[@subType="ACTUAL"]}
to \icode{FindByComponentAndName}.
So, for example, the following call would extract only commanded position
measurements for \emph{any} Linear axis:

\begin{quotation}
  \icode{evts = resp.FindByComponentAndName("Linear", $\neg$ \\
    String.empty, "//[@subType='COMMANDED']");}
\end{quotation}

Note the use of single quotes inside the double quotes to delimit the
constraint string.  We could also have used the standard C\# syntax for
including quotes in a string:
\icode{"//@subType=$\backslash$"COMMANDED$\backslash$"}.

Finally, if you want to just specify an XPath expression directly to
capture all of the constraints, you can use \icode{FindByXPath} to do
just that.  You can pass it a string or a precompiled XPathExpression
object 
(this is an advanced feature of the .NET support for XPath, so we'll
just illustrate using a string.)  The following call to
\icode{FindByXpath} also returns only commanded position measurements
for Linear axes (compare it to the example above):

\begin{quotation}
  \icode{evts = resp.FindByXPath("//Linear//[@subType='COMMANDED']");}
\end{quotation}

\subsubsection{Discussion}

In our simple example, we didn't pass any arguments to Current. As you
recall from section~\ref{sec:cmd:current}, this causes Current to return
a ``complete snapshot'' of all reportable measurements, even though our
simple app is then only selecting a small subset of them to examine.

Because our example machine is pretty simple, this is probably OK.  But
if we were dealing with a more complex production machine that can
report hundreds of measurements at a time, it's inefficient to ask for
everything just to look at one or two items.  In that case, we'd be
better off passing a Path argument to Current, to narrow down which
measurements will be returned.  An \gloss{overloaded} version of the
\icode{Current} method of \icode{MTCAgent} allows us to pass an XPath
expression directly to constrain ``up front'' which measurements are
reported.  To illustrate this technique,
example~\ref{fig:code:helloworld2} shows an alternative version of our
simple client in which we only request the Power subsystem's
measurements when we do the Current command; then when we call
\icode{FindByComponentAndName} we just pass empty strings for both
\icode{comp} and \icode{name}, since we know there's just one set of
measurements in the reply anyway.


\subsection{Example 2: Working With Samples}
\label{sec:example-samples}

In this section we show example code for working with larger volumes of
samples at once.  Rather than just using the Current command to get
subsequent ``snapshots'' of the machine state, we'll use the Sample
command to periodically download a batch of samples and export it for
use by other applications.


\label{sec:example-file}

In this section we show example code to capture some data according to
user specifications and then write that data into a CSV (Comma-Separated
Values) file.  Many popular tools, including databases such as Microsoft
Access or MySQL and spreadsheets such as Microsoft Excel, can import CSV
files directly.

\subsection{Example 4: Alarm Monitoring}
\label{sec:example-alarm}

TBD
