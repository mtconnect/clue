\section{Introduction}

\subsection{What Is \mtc?}

\mtc\footnote{MTConnect is a trademark of the
  Association for Manufacturing Technology.  While this document and the
  software described in it are licensed  royalty-free as described in
  section~\ref{sec:license}, use of the MTConnect trademark must be
  approved by The Association for Manufacturing Technology, as described
  on \website.}
is a standard that codifies a simple  \emph{command
  set}  and \emph{data representation} for the easy exchange of
manufacturing data.  White papers and overview documents describing
\mtc{} conceptually can be found at \website.

\mtc{} is a lightweight standard because it is built on top of mature
open standards that are functionality-rich, widely deployed, understood
by many programmers, and supported by many development tools.
Both \mtc and all the standards on which it depends
are unencumbered from an intellectual property standpoint:
specifications can be freely downloaded by
anyone (though this isn't necessary in order to successfully use \mtc),
and no royalties are required in order to implement the standards.  See
\website{} for more detailed information on licensing.

\subsection{Purpose and Scope of This Document}

Since \mtc{} is based entirely on 100\% open standards, \textbf{no SDK}
is necessary to create \mtc{} applications.  Nonetheless, AMT provides a
set of Client Libraries, Utilities and Extensions (CLUE) to facilitate
``getting you off the ground'' with simple applications that wish to
avoid dealing directly with HTTP and XML altogether.  The current
version of CLUE, described in this tutorial, works with the Microsoft
.NET framework.  .NET is \emph{not} an open standard and
currently runs only on Windows, but we chose to support it because of
its wide popularity and customer demand.  The CLUE libraries themselves
are released under the same liberal open-source license as all other
\mtc{} software; only .NET itself is proprietary, not the
.NET-compatible code provided in the form of the CLUE libraries.

\subsubsection{Goals}

\begin{itemize}
\item A ``quick start'', tutorial-style guide
for engineers wishing to write applications
that collect and analyze data from \mtc-compliant equipment.  
\item An overview of both the \mtc{} ``native'' API's (using XML and
  HTTP) and the \mtc{} Client Utility Library for Microsoft .NET
  Framework.
\end{itemize}

\subsubsection{Non-Goals}

\begin{itemize}
\item  \textbf{Not a specification:}  The definitive
MTConnect specification can always be found at \website.  
\item \textbf{Completeness:}
While all the
examples presented are correct and \mtc-compliant
at the time of writing, we do not use or describe every
feature of \mtc{} here, nor do we guarantee that these
examples will remain correct as the specification evolves.
\item The CLUE library is built on top of .NET's excellent support for
  HTTP, XML and XPath.  This document does \emph{not} describe how to
  write applications using those underlying libraries directly.  You are
  encouraged to browse the heavily-commented (and open source) CLUE code
  in addition to the wealth of available literature on programming XML
  and XPath using .NET.
\item This guide does \emph{not} explain how to design \mtc-compliant
  controllers or create adapters to make legacy controllers
  \mtc-compliant. The separate
\mtc{} Implementor's Guide covers those topics.
\end{itemize}

\subsubsection{Assumptions and Prerequisites}

The intended audience of this document should:
\begin{itemize}
\item Have basic knowledge of using Web applications
\item Have experience developing code in one or more modern programming
  languages or frameworks (C/C++, C\#, Microsoft .Net, Java, etc.)
\item Understand the basic concepts of Object-Oriented Programming (OOP)
  at the level of a language such as Java, C++ or C\#.
\end{itemize}

\subsubsection{Topics Covered}

\begin{enumerate}
\item Section~\ref{sec:arch} describes the architectural concepts and
  components behind \mtc{} and should be read by everyone who wants
  to use \mtc{} in any way.  It describes 
  how to construct and transmit
  commands to an \mtc{} Agent and how to interpret the responses to
  those commands encoded in XML (eXtensible Markup Language), all using
  an ordinary Web browser.
\item Section~\ref{sec:apps} describes how to write client applications
  for \mtc{} using the AMT-supplied Client Library, Utilities and
  Extensions (CLUE) for Microsoft .NET.  The CLUE libraries are not part
  of the \mtc{} specification, but they streamline the task of writing
  simple \mtc{} applications.
\comment{
\item Section~\ref{sec:discovery} describes how a client application can
  automatically discovery \mtc-compliant devices using the optional
  discovery capabilities of \mtc{} via LDAP, the Lightweight Directory
  Access Protocol (a/k/a Windows Active Directory).  Not all
  \mtc{} installations  will include discovery functionality, so it is
  not a required part of the specification.
}
\end{enumerate}



\subsection{License, Copyright and Trademark Information}
\label{sec:license}

\subsubsection{Software License}

You are free to use the CLUE libraries and any AMT-provided example code
in your own commercial or
noncommercial products without paying any royalty or notifying AMT, as
long as you acknowledge AMT's copyright on the underlying code, agree
not to hold AMT responsible for any aspect of the code's behavior, and
do not use the name(s) of AMT to endorse or promote your products
without AMT's express permission.

(Keep in mind that there are many, many open source implementations of
HTTP, XML and XPath libraries available for a variety of platforms; you
can also use any of those in lieu of CLUE.)

The following are the terms of the license (also available at \website),
as of this printing
(Fri, 07 Dec 2007)
:

\plaintextfile{license.txt}

The following trademarks, registered trademarks, or trade names appear
in this documentation for identification purposes only, and do not
indicate any affiliation with nor endorsement by The Association for
Manufacturing Technology or the \mtc{} Consortium.

Visual Basic, Visual Studio, VB.NET, .NET Framework, 
C\#, J\#, Microsoft Excel, Microsoft Windows, Microsoft Active Directory,
Microsoft Internet Information Server (IIS), Microsoft Access,
and Microsoft Internet Explorer are trademarks of Microsoft Corporation.  
Firefox is a trademark of Mozilla.  Safari is a trademark of Apple Inc. 
% Fanuc is a trademark of GE~Fanuc~Automation.  
% ArchRock is a trademark of ArchRock Corporation.  
IPC-CAMX is a trademark of IPC--Association Connecting Electronics Industries.
MySQL is a trademark of MySQL~AB.

% \subsubsection{Document License}
