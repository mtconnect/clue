
\section{Automatic Discovery Using LDAP}
\label{sec:discovery}

So far, all the examples have assumed that the client application knows
the base URL at which to contact the Agent
(e.g. \icode{http://agent.mtconnect.org/LinuxCNC}).
In some applications, this may not be known in advance.  For example, an
application might be written to automatically locate (discover) all
\mtc-enabled machines in the environment and synthesize a simple UI for
monitoring their power and controller states.

To accommodate these scenarios, the \mtc{} specification includes
functionality for a client to \emph{discover} the existence of any
Agents in its environment.  The protocol used for discovery is
\gloss{LDAP}, the Lightweight Directory Access Protocol (also known in
the Windows world as Active Directory), designed as a simple way for
clients to query a hierarchical directory of information.  LDAP supports
authentication, that is, the LDAP server may require the client
application to supply a password or other credentials in order to be
allowed to look things up.

The following three caveats apply to our brief description of discovery
in \mtc:

\begin{enumerate}

\item 
As with XPath, LDAP is quite general but we will focus only on the small
subset of its capabilities required to illustrate how it \mtc{} is
supported.  For more detailed information, including information on
administering an LDAP server, we suggest consulting one of the many
available publications on Windows Active Directory.

\item Also as with XPath, the CLUE libraries
shield simple applications from dealing directly with LDAP.
We'll focus on examples that take advantage of this functionality, but
.Net's technical documentation describes its excellent LDAP support
should you wish to do more.

\item Keep in mind that the use of LDAP is entirely optional in \mtc{}.
  \mtc{} specifies how Agents should appear in an LDAP directory
  \emph{if} such a directory is available and \emph{if} the
  administrator of the installation chooses to list the Agents  in it;
  neither  of these are required by the specification.
\end{enumerate}


\subsection{LDAP In a Nutshell}

Which LDAP server? or default

Which attributes must be specified?


\subsection{LDAP Support in CLUE}
\label{sec:ldap-clue}

In section~\ref{sec:app} you saw how to create a new \mtc{} Agent object
by giving the URL of an Agent.  CLUE also provides a class-level method
called \icode{FindByLDAP} that will construct and send the necessary
LDAP query to automatically find any Agent(s) in the environment.  It
returns an array of zero or more \icode{MTCAgent} objects.  If the
array has length zero, it means no Agents could be found via LDAP, and
you can look at the class-level variable \icode{errors} for a
human-readable description of why not.

There are three versions of \icode{FindByLDAP}.  The simplest one takes
no arguments and uses the default LDAP server (see
section~\ref{sec:ldap-intro}).  The second version takes the hostname of
the LDAP server to use, and contacts it on the standard LDAP port (TCP
port 389).  The third version allows you to also specify a port number
in case the LDAP server is running on a nonstandard port.

Here's an example of locating an agent via LDAP:

\listing{figures/helloworld_ldap}{fig:code:helloworld_ldap}{
  Locating an Agent using the CLUE support for LDAP.
}


\boxednote{Why doesn't AMT provide a reference LDAP Server?}{LDAP
  servers are typically part of an organization's IT infrastructure and
  LDAP servers are already included with most enterprise IT systems
  including Windows and Linux server systems.  Configuring an LDAP
  server and populating it with \mtc information is typically in the
  purview of IT staff and is closely controlled, so there's no point in
  providing a reference LDAP server.}


