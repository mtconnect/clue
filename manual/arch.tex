
\section{Architectural Review of \mtc}
\label{sec:arch}

In this section we review the key architectural concepts of \mtc.  It's
strongly suggested that you follow the examples by using a Web browser 
to interact with the live
Agent provided by the \mtc~Consortium for developer testing.  In the
next section we will see how to interact with the agent from a Client
program rather than ``manually'' via a Web browser.

\subsection{\mtc~Environment}

The \mtc specification requires that all \mtc communication take place
over \gloss{HTTP}\footnote{When the name of a protocol or technology
  appears in this typeface, it means you can look in the Glossary
  (section~\ref{sec:glossary}) for a concise
  description of it.}
running over \gloss{TCP/IP}.  Therefore,
Clients can be developed in any programming language that has
support for networking, and you can even use a Web browser (which is
basically a generic HTTP client) to interact in limited ways with \mtc{}
devices. 

An \mtc~Agent is a software entity that interprets \mtc{} commands and
returns machine data on behalf of one or more machines that it
``represents'' in the \mtc{} environment (hence
the term Agent).  The Agent software may be part of the
controller's software stack, or it may be a separate piece of software
running on a computer that communicates with a controller via a
proprietary interface.
The \mtc~Consortium operates
a publicly-accessible simulator of an \mtc-enabled vertical CNC
mill\footnote{The simulator is based on the \gloss{open 
  source} controller EMC2 (\url{http://www.linuxcnc.org}).}. 
If you have access to a Web browser, it's strongly recommended you use
it to contact the Test Agent and follow the examples in this guide.

\boxednote{You mean Agents are publicly accessible?}{We've chosen to
  make the Test Agent publicly accessible with no passwords, firewalls
  or other security measures, for the convenience of
  \mtc~developers. In a production scenario, the Agent would likely be
  protected by one or more of these standard security measures.}

\mtc{} data communicated from an Agent to a Client is represented using
\gloss{XML}, the eXtensible Markup Language, which provides for a
human-readable, ASCII-based representation of hierarchical data.  The
specific XML constructs permitted in \mtc are codified using the
\gloss{XML~Schema} standard, but you don't need to be familiar with
XML~Schema to use \mtc.  Of course, data pulled from machines can be
stored in a non-XML format as well---for example, it can be stored in a
\gloss{relational database} or imported into an Excel spreadsheet.  But
all data communicated between an \mtc{} device and its client
applications must be represented in XML.

Besides the Agent and the Client, there are two more basic concepts you
must understand to get started with \mtc:

\begin{enumerate}
\item Device schema: how a device, such as a controller attached to a
  machine tool with various axes and fixtures, describes what data are
  available to collect.
\item Data representation: the format in which  result data
 is delivered to the requester.
\end{enumerate}

\mtc~1.0 defines one command, \icode{probe}, that allows a client to
discover the device schema of the machine represented by a particular
Agent; and two commands, \icode{sample} and \icode{curren}, to actually
collect data.

\subsection{The Probe Command and the Device Schema}

The \mtc{} \icode{probe} command is used to interrogate what data an
Agent is able to report on behalf of the machine(s) it represents.
(The most common case
is for an Agent to represent exactly one machine, but it is possible for
a single Agent to represent multiple machines.)  

Each machine represented by an Agent is called a \emph{device} in \mtc.
Therefore, the response to a probe is a \gloss{tree} of components,
whose ``root'' or 
topmost node, Devices, has \emph{children} 
(nodes directly connected to it and one level lower in the hierarchy)
representing each  individual machine (Device).  
In other words, you can think of the Devices
node as a ``container'' that groups together a logical collection of
devices---for example, a cell on a shop floor---whereas each Device node
would represent a particular piece of hardware.  

\subsubsection{Issuing the Probe Command}

To issue any \mtc{} command to an Agent, you construct a \gloss{URL} by
taking the base URL of the agent (in our example, the Test Agent is
running at
\icode{\agenturl} and appending a slash followed by the command name (in
this case, \icode{probe}).  In later sections we'll see how to pass
arguments to the command, but the \icode{probe} command takes no arguments.

So, point your Web browser at \icode{\agenturl/probe} and you will see
the device schema of a simulated vertical CNC mill, which is programmed
to repeatedly cut a spiral pattern using the simple GCode program shown in
listing~\ref{code:spiral}.  You should see something very similar to
listing~\ref{fig:probe-output}; in the listing we have broken up the
lines to improve readability, and we have numbered the lines for
reference. 

\subsubsection{What Can Be Reported?}
\label{sec:interpreting-device-schema}

What data can this test Agent report?
The beginning of the Devices
``container'' contents is indicated by the \gloss{XML tag}
\verb+<Devices>+ (figure~\ref{fig:probe-output}, line 9) and the end of
the container contents by \verb+</Devices>+ (line 85).  This particular
Agent represents only a single Device (the simulated LinuxCNC vertical
mill controller), so the Devices node encloses only a single Device
node, which starts at line 10 and ends at line 84.

The children of each
Device (i.e., the entities contained ``within'' it) 
are its subassemblies or components; the \mtc spec states that
\emph{every} Device
must have at least a Power component, which at a minimum reports whether
the device is powered on or off.   You can see at lines 77--82 that
there is indeed a Power component.

You can also see that there are four axes that can deliver
measurements.  One is a spindle (lines 23--32), and there are also
linear X, Y and Z aces (lines 33--42, 43--52, 53--62 respectively).
Notice that each Linear axis has a different name (``X'', ``Y'' and
``Z'' respectively); this shows that a given component can have multiple
subcomponents of the same type (Linear axis) as long as they all have
different names.

\subsubsection{DataItems: What Actual Data Will Be Delivered?}

What specific data---in what units, etc.---will be reported for each
component?  Take a look again
at lines 33--41, which describe the Linear X axis measurements
available.  Inside the Linear node, we find a list of DataItems.  Each
item in this list corresponds to one type of measurement that is
available for this component (the Linear X axis).

Lines 35--37 and 37--39  show the two types of measurements that this
axis can report.  Both measurements are of type POSITION (contrast with
the Spindle type axis, whose measurement type is SPINDLE\_SPEED, as you
can see in line 25).  One measurement corresponds to a COMMANDED
position and the other to an ACTUAL position (the subType attribute of
each DataItem).  Both measurements are delivered in units of MILLIMETER,
and both are in the category SAMPLE.  The category tells you, in effect,
what ``kind'' of data this is:  a numerical sample value, an event (such
as an alarm or a power state such as ON vs. OFF), and so on.  For
example, you can see that the Power subsystem (lines 77--82) can deliver
a ``measurement'' whose type is POWER\_STATUS and whose category is
EVENT (rather than SAMPLE).  The reason for this is that SAMPLEs are
used to denote actual numerical data sample values, whereas EVENTs are
used to denote properties that can only take on one of a fixed number of
values, such as ON or OFF for the POWER\_STATUS.  

\boxednote{What determines whether a component reports SAMPLES or
  EVENTS?}{The complete \mtc{} specification spells out exactly what 
  different categories, types, and subtypes are allowed to appear in the
  response to a \icode{probe}.  The \mtc{} Specification document
  describes these in plain English, but the authoritative (and most
  precise) description is the XML~Schema file available from the \mtc{}
  Web site, \website.}

You've no doubt noticed that each component has other attributes we
haven't mentioned---for example, fields such as \emph{id} or
\emph{nativeUnits}.  You can read about all of these in detail in the
\mtc{} specification; in general we'll introduce them only as needed to
get you writing applications quickly.  However, it is worth noticing
some attributes describing the Device represented by this agent (see
line 10), because they tell you something about the machine:

\begin{itemize}
\item \verb+uuid+ is an identifier that uniquely identifies this
  component.  A common method for generating these is to reverse the
  Internet domain name of the organization (e.g. mfg.mycorp.com becomes
  com.mycorp.mfg) and then adding the component's serial number or other
  unique identifer (e.g. com.mycorp.mfg.2205178).
\item \verb+name+ is a human-readable name describing the component,
  LinuxCNC in our example.
\item \verb+sampleRate+ is the maximum rate at which this device
  samples, in samples per second.  We'll return to this when we use the
  Sample command.
\end{itemize}

As you can see, \icode{probe} doesn't actually return data samples
values; it only returns information describing the \emph{structure} of
the device and component hierarchy.  Next we'll use this information to
determine what data to request.

\subsubsection{Paths and XPath}
\label{sec:xpath}

The hierarchical arrangement you saw in the output of \icode{probe}
leads us to the very important concept of 
the \emph{path} to a node in the hierarchy.  You can think of a path as
a description of how to traverse the data, starting from the root node
(the node labeled \emph{m:MTConnectDevices} in line 2 of
figure~\ref{fig:probe-output}; don't worry too much about the name) and
ending up at the node or nodes with the measurements you are actually
interested in.

For example, to ``navigate'' to the node describing the Z-axis Actual
position in figure~\ref{fig:probe-output}, you might describe it like
this:

\begin{enumerate}
\item Begin at the root node, \emph{m:MTConnectDevices} (line 2).
\item From there, look for the Devices child (line 9).
\item From there, look for the first Device child (line 10).
\item From there, find the Components associated with the Device (line
  15).
\item From there, locate the Axes subcomponent (line 16).
\item From there, find the Components list that makes up the Axes (line
  22).
\item Within that list, find the Linear component whose name is Z (line 53).
\item Finally, inside that component, you can see what data items it
  will return.
\end{enumerate}

Using the same ``directory'' notation you've seen for navigating file
systems, we could write down the \emph{path} to the Z axis data as:

\begin{quotation}
\icode{/m:MTConnectDevices/Devices/Device/Components/$\neg$ \\
  Axes/Linear[@name="Z"]/DataItem[@subType="ACTUAL"]}
\end{quotation}

(where we've used the symbol $\neg$ to indicate a line break that
implies  no additional spaces, i.e. the above string should be all run
together on one line).  In the above expression, the square brackets
indicate additional \emph{restrictions} in following the path; for
example, the excerpt \icode{Axes/Linear[@name="Z"]} can be read as
``From the Axes component, find the child called Linear \emph{whose name
  is `Z'}.''  In the case of Devices and Device, we didn't need these
extra qualifiers, since there is only one Devices node and it has only
one Device node as a child.

The notation above is legal syntax for \gloss{XPath}, a standard for
describing how to ``navigate'' an XML document (such as the Probe
output) to find what you want.
However, it's also somewhat cumbersome.  We can make it much more
readable and compact by taking advantage of the fact that XPath doesn't
require us to navigate ``step by step''.  In particular, above
we used the notation \icode{A/B} to indicate ``from A, navigate to the child
of A that is called B'', but XPath also lets us use the notation
\icode{A//B} to indicate ``from A, navigate to the \emph{descendant} of
A that is called B.''  As in real life, a descendant could be a child, a
grandchild (child of a child), etc.  So it should be easy for you to
convince yourself that the following XPath expression will also take us
to the Z-axis Actual measurement:

\begin{quotation}
  \icode{//Devices//Device//Axes//Linear[@name="Z"]}
\end{quotation}

Not nearly as bad!  Notice that \icode{Devices//Device} is just as valid
as \icode{Devices/Device}: since Device is a child of Devices, by
definition it's also a descendant of Devices.  Note also that we can
omit the root node:  since the path starts with \icode{//Devices}, it
implicitly means ``start at the root of the tree, and locate the first
descendant called Devices.''  In fact, for our simple CNC controller,
even the following path would be valid:

\begin{quotation}
  \icode{//Linear[@name="Z"]} 
\end{quotation}

This says: ``From the root of the tree, find the first descendant that
is a Linear node whose name is `Z'.''
This works because we know that in our simple machine, 
only one possible measurement can match that criterion.  On the other
hand, if we just gave the path as \icode{//Linear}, it would match all
three of the Linear axes (named ``X'', ``Y'' and ``Z'' respectively).

Next we'll use this path notation to actually get data samples.

\subsection{Sample and Current: Data Items and Events}
\label{sec:measurements}

An \emph{event} is the act of capturing a data item; in addition to the
raw data value, the event includes metainformation such as the sequence
number of the event (showing the order in which the measurements were
taken or alarms noted) and the time at which the event occurred (when a
measurement was taken or an alarm noted, to whatever time resolution is
supported by the device).

% Developers accustomed to the RPC-style (Remote Procedure Call)
% programming pattern, in which each request made by a client has exactly
% one corresponding response from the ``server'', may wonder why \mtc
% bothers with the notion of an ``event'' rather than a simple
% request/reply model.  Under the RPC pattern, the Client in essence says
% ``Take a snapshot of your state, and report it to me.''  

The reason for Events is that \mtc{} tries  to avoid assumptions
about the relationship between the rate at which a particular controller
produces  data and the rate at which
a particular client application wants to consume it.  As a simple
example, consider spindle load.  A particular Agent (embedded in a
controller, say) may be able to produce samples of this value every
10 milliseconds, and indeed a sophisticated monitoring client
might want to collect such fine-grained samples.  But what about a
coarser-grained monitoring client that only collects a couple of
``snapshots'' per second of this data?  How would the Agent
``know'' at what granularity it should be capturing samples?

Using Events eliminates this apparent problem.  You can
think of the Agent as ``filling a bucket'' with data samples (events)
on its own, and different applications then ``draw samples'' from the
bucket at different frequencies as needed.  Periodically, as the bucket
gets full, the Agent discards the oldest samples---whether
anyone has seen them or not.

\subsubsection{The Current Command: Show Me a Snapshot}
\label{sec:cmd:current}



The command \icode{current} returns the \emph{most recent} set of measurement
samples from all reporting components.  It takes an optional argument,
but we will first explain its behavior when this optional argument is
omitted. 

You already saw that a command is transmitted to an Agent by
constructing an HTTP URL consisting of the Agent's base address and the
name of the command.  So, point your Web browser at
\texttt{\agenturl/current}, and you should see output similar to that
shown in figure~\ref{fig:current-output} (the timestamps and some of the
values may be different, and we made the formatting easier to read in
the printed listing, but the structure should be the same).

Let's see how the output of \icode{current} in
figure~\ref{fig:current-output} corresponds to that of \icode{probe}.
Recall that when we looked at the output of the \icode{probe} command
(figure~\ref{fig:probe-output}), we looked at the Axes entity (line 16)
which has a child Spindle (line 23) whose name is ``S'', and that
Spindle is capable of reporting its \x{ACTUAL} \x{SPINDLE\_SPEED} (line
25) in units of \x{REVOLUTION/MINUTE} (line 28). While you're there,
notice the \emph{id} of this DataItem is 10 (line 27).

Now look at line 36 of the output from the \icode{current} command
(listing~\ref{fig:current-output}), 
which begins a \x{ComponentStream} for the device identified by the
path \x{/Device[@name="LinuxCNC"]//Axes[@name="Axes"]} and the specific
component \x{S} within that path.  The correspondence makes it clear
that the measurements contained in this \x{ComponentStream} element
correspond to that same spindle.

What are the values of those measurements?  You can see a \x{Samples}
element (line 39) 
containing a single sample (line 40),  a \x{SpindleSpeed}
measurement whose value is 3400.0.   The measurement was taken at
11:18 and 16.372 seconds on May 23, 2008 (the \x{timestamp} attribute of
the \x{SpindleSpeed} sample).  


What are the units of the measurement?  The fact that the SpindleSpeed 
measurement also has an \emph{itemId} of 10, which corresponds to the
\emph{id} of 10 in line 27 of the \icode{probe} output, is a further way
of telling you that this SpindleSpeed sample is a measurement of the
type described by the DataItem structure in that \icode{probe}.  That
DataItem tells us that
the units 
are revolutions per minute (figure~\ref{fig:probe-output}, line 28).
In other words, we looked at the \emph{itemId} of the measurement value
from \icode{current}, and matched it up with the \emph{id} of the
DataItem from  \icode{probe} to determine, in effect, how to interpret
the value.

There's another way (besides matching up the \emph{itemId} and \emph{id}
fields) to do match these up as well.  Notice the \x{path} attribute of
the \x{Samples} element (figure~\ref{fig:current-output}, line 37).  You
can easily verify that the path given there, namely 
\texttt{/Device[@name='LinuxCNC']//Axes[@name='Axes']}, is an XPath
expression.  If we apply this path to the Probe output
(figure~\ref{fig:probe-output}), 
just as we did in section~\ref{sec:xpath},
it would return the entire \x{Axes}
node, which spans lines 11--42 in figure~\ref{fig:probe-output}.  

And as we'll see, the Client Libraries for \mtc{} automate much of this
matching-up and document-traversal for you.

\subsubsection{Test Your Understanding\ldots}

In a similar way, you should be able to look at
figures~\ref{fig:probe-output} and~\ref{fig:current-output} and convince
yourself of the following:

\begin{enumerate}
\item The Z-axis is a Linear axis that supplies Position measurements in
  units of millimeters; it is capable of providing both a commanded and
  actual position measurement.  (Probe, lines 53--62)
\item The most recent actual value of the Z-axis is $-0.1000000015$mm,
  although the commanded value was $-0.1$mm.  (Current, lines 28--33)
\item The Power subsystem (Probe lines 77--82) is only able to report
  Power status.  Power status is an Event, rather than a Sample, because
  it is not a measured value but one of a finite number of possible
  states.  
\item At the time this sample was taken, the current Power
  ``measurement'' shows that the power state is ON (Current lines
  48--52).
\item The Controller subsystem (Probe lines 65--76) is able to report
  various items including Line, Controller Mode, Program and Execution.
  However, the current sample (Current lines 54--68) only reports values
  for Controller Mode (Automatic), Execution (Executing), and Program (a
  GCode file called \texttt{spiral.ngc}; it doesn't report a value for
  Line.  This could be because this particular controller doesn't report
  that value.  It illustrates the important concept that while an entity
  (e.g. Controller) may be \emph{logically capable} of reporting on a
  particular type of measurement, that doesn't necessarily guarantee
  that the measurement \emph{will always} be reported given some
  specific machine. 
\end{enumerate}

As you can see from the above discussion, the \icode{current} command
gives you  the \emph{data}
associated with one or more measurements, and each data item can be one
of a number of different types of data---a Sample (which itself can be a
Position or a SpindleSpeed in this example) or an Event (such as the
Execution, Controller Mode and Program ``measurements'' associated with
the Controller in our example).  The complete specification of 
different types of events and samples is codified in the \mtc{} XML
Schema.

You also saw that the \icode{probe} command gives important
\gloss{metadata} (``data about the data'') associated with
measurements.  For example, it is the output of the Probe command that
tells us that  \x{SpindleSpeed} is reported in RPM whereas the \x{Z}
axis position is reported in millimeters.

\subsubsection{The Current Command with a Path Argument}
\label{sec:cmd:current-with-path}

In the previous example, we didn't pass any arguments to the
\icode{current} command, so by default it reported current values for
all measurements reportable by this Agent.  However, you can also call
\icode{current} with an argument that restricts its attention only to
certain measurements.  To do this, we simply construct an XPath
expression (similar to what we saw in section~\ref{sec:xpath}) that
matches only the parts of the ``tree'' we care about.

For example, suppose we care only about the feedrates of the various
axes, and not the controller program state.  If you pass an argument
called \emph{path} to the \icode{current} command, the samples returned
will be restricted only to those matching the given path.  Referring
again to the output of Probe (figure~\ref{fig:probe-output}), recall
that the path \texttt{//Axes} will match only the Axes element of the
schema.  

To issue a command with arguments to an Agent, we build up the command
URL as usual, and then
append a question mark after the command name, followed by the
argument name and value:

\begin{quotation}
\texttt{\agenturl/current?path=//Axes}
\end{quotation}

Type this into a Web browser and you'll see that the result you get is
much more concise than what you got when \icode{current} was issued with
no arguments.  Indeed, what you get is that subset of the full snapshot
that matches the path \icode{//Axes}.

Similarly, the path \texttt{//Axes//Spindle} will match only the Spindle
axis and not the other (linear) axes; you can read this path as: ``Find
all the Axes components, wherever they occur; then, find any Spindle
components that are descendants of those Axes components.''

What if we wanted only the linear Z axis?  You might think to try
\texttt{//Axes//Linear}, but all three (X, Y, Z) axes  match that path.
In this case, we  can
constrain the match using the \emph{attributes} of an element, as we
showed in the first example in section~\ref{sec:xpath} 
and write the path expression \texttt{//Axes//Linear[@name="Z"]} to get
what we want.  You can read this as:  ``Find
all the Axes components, wherever they occur; then, inside each one of
them, find any Linear components \emph{whose \emph{name} attribute has
  the value `Z'}.''  Recall that we don't need to use this trick to select the
Spindle, because there is only one Spindle; but it would be
perfectly fine to use the path \texttt{//Axes//Spindle[@name="S"]} in
that query, leading to the URL

\begin{quotation}
  \icode{\agenturl/$\neg$ \\
    current?path=//Axes//Spindle[@name="S"]}
\end{quotation}

What if the path we give doesn't match anything?  For example, what if
we said

\begin{quotation}
  \icode{//Axes//Spindle[@name="NoSuchSpindle"]} or \\
  \icode{//Axes//NeitherLinearNorSpindle}  
\end{quotation}

If you try it, you'll see
that you still get back a well-formed XML response document---but it
will contain no measurements or data samples.  In particular, the
document will still have a Streams element that contains a DeviceStreams
list, but that list will be empty.  This illustrates that it's not an
error for an XPath not to match anything.

\boxednote{Try It}{You're strongly encouraged to try these behaviors
  yourself by typing these URL's right into your browser.  There's no
  better way to get familiar with the XPath syntax than doing it
  yourself.}

XPath syntax is extremely rich, and if you know how to use it, you can
construct quite sophisticated queries.  We will stick to 
simple examples, but one interesting syntax construct is the use of
the vertical bar or ``pipe symbol'',  \verb+|+, to mean ``one or the
other.''  You construct each path expression completely separately, then
join them with \verb+|+.  Thus in the above example, if we wanted the
values of the X and Y axes only, we could use:

\begin{quotation}
  \icode{\agenturl/$\neg$ \\
    current?path=//Axes//Linear[@name="X"]|//Axes//Linear[@name="Y"]}
\end{quotation}


\subsubsection{The Sample Command}
\label{sec:cmd:sample}

The command \icode{sample} is a generalization of \icode{current}.
Rather than returning only the most recent measurement sample from one
or more components, it returns a set of samples.

It takes up to four arguments. 
The \icode{start} argument tells the Agent what the first (earliest)
measurement to be returned should be.  Notice in the example output from
the Current command (figure~\ref{fig:current-output}) that each sample
item has a sequence number associated with it (look for the
\emph{sequence} attribute associated with each \emph{m:Position}
measurement, for example at lines 15, 19, 29, etc.)  
Every time an Agent records a new event from a controller, it
increments the sequence number associated with that controller, so that
every single measurement taken gets its own sequence number.  The
sequence numbers are a simple way to number the events so you can tell
which ones you've seen before.  

Thus, the \icode{start} argument to the \icode{sample} command is the
desired starting sequence number of data items.  The Client must supply
this since there may be multiple Clients communicating with a single
Agent, and the Agent doesn't necessarily keep track of which sequence
numbers have been seen by which clients.  However, looking again at the
output of the Current command, you'll see that the Header element has
the attribute \texttt{nextSequence="52056912"} (line 8).  This is the
Agent's way of telling the client what sequence number that client
should request next, i.e. it's the lowest sequence number that client
hasn't seen yet.  As you'll see in the next chapter, the Client Utility
Library keeps track of this for you automatically.

\boxednote{Why don't I have to keep track of the sequence number for the
  Current command?}{By definition, the Current command returns the
  ``most recent'' sample value for one or more measurements---so you
  don't need to supply a sequence number.}

The optional \icode{count} argument specifies a maximum number of samples to
return, and defaults to 100 if no value is given.  Fewer samples may be
returned if not enough are available (for example, 
if \icode{count} is 100, but fewer than 100 samples have been
reported since the desired starting sequence number).  In fact, if there
are \emph{zero} samples available matching the given path and
starting sequence number (perhaps the machine is offline or idle), zero
events may be returned.

The last argument, \icode{freq}, is for
scenarios in which a Client wishes to continuously receive a streaming
data feed of specific data items.  Use of this argument is optional
and we won't cover it further in this
overview, since it requires some programming techniques not covered in
this tutorial.

As before, we encode these arguments into the URL by appending a
question mark, and we separate multiple arguments with ampersands (\&).
For example, to issue the Sample command with arguments
\emph{count$=100$}, \emph{sequence$=23456$} and
\emph{path}=``//Axes//Linear'', we would construct  the URL

\begin{quotation}
  \icode{\agenturl/sample?count=$\neg$ \\
    100\&sequence=23456\&path="//Axes//Linear"}
\end{quotation}

In the unlikely event you need to embed an ampersand in the URL as
part of an argument value, you can
use the escape sequence \verb+&amp;+ (you need both the \& and the
trailing semicolon) to do so.)

\subsubsection{Recap: Things to Remember}

\textbf{``Error'' responses are rare:}
Keep in mind that as long as a command to the Agent is
well-formed---i.e., the Agent URL is correct, values have been supplied
for required arguments, and any values for optional arguments are
legal---then the response will \emph{always} be a valid XML document
conforming to the \mtc{} XML Schema.  Of course, in some cases, the
document may not have any ``interesting'' content: as we saw above, if
you specify an XPath that doesn't correspond to any component that can
deliver measurements, or if you request samples but there are no new
samples to report, you may get back a document with zero samples.  But
it is still a legal response and must still conform to the \mtc{} XML
Schema.  Of course, if the URL encoding the command to the Agent is
ill-formed, you  may get an HTTP error.

\textbf{Is Probe always necessary?}
A Client can certainly issue \icode{sample} or \icode{current} commands
without ever having issued a \icode{probe} first.  For example, a
particular Client may be designed to work only with a specific 
device, and the device's schema may be ``hardwired'' into the
Client code.  For example, we used knowledge from the
Probe command above to determine in what units a particular measurement
was reported.  But if the engineer writing the application ``knows''
what units and measurements are reported by the machine, and he's
writing an application that will only gather data from that specific
machine, he may choose to ``hardcode'' this knowledge into the
application itself rather than going through the extra step of doing a
Probe.  Of course, if the controller behavior changes for some reason,
this would be reflected in the output of the Probe command; so it's more
portable and somewhat safer to use Probe rather than hardwiring this
knowledge into the application.

In contrast, a more generic Client, such as a
monitoring/analysis tool that can be used with a variety of devices,
might first do a \icode{probe} to find out what devices are available
to be monitored, possibly presenting the human operator with a menu or
selection list to narrow down what she wishes to see, and would also use
the Probe output to get information about the units in which each
measurement is reported, in order to perform unit conversions.

As we'll see, the Client Libraries described in the next section
\emph{do} expect you to do a Probe first, because they will then
remember for you such information as measurement units and information
about the machine itself.


