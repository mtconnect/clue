\section{Technology Glossary}
\label{sec:glossary}

\mtc{} is built on the following completely open technologies.  While
you do not need a detailed understanding of any of them in order to use
\mtc{}, they are mentioned here for reference and presented in logical
rather than alphabetical order.

Throughout the article, references to standards or
technologies described below appear \gloss{like this}.

\begin{itemize}
\item \textbf{HTTP} Hypertext Transfer Protocol, the protocol used to transfer 
  data between Web servers and Web browsers.  This protocol, and most of
  the other Web-related standards used 
  by \mtc, are stewarded by the World Wide Web Consortium 
  (W3C), \url{http://www.w3.org}.
\item \textbf{LDAP} Lightweight Directory Access Protocol, the protocol used to
  query a directory server (also known in the Windows world as Active Directory)
\item \textbf{Open Source} Sometimes abbreviated FOSS (Free and Open Source
  Software).  Any software that is available free of charge and
  usually licensed under terms that require no royalty payment for its
  use, redistribution or modification.  The GNU General Public License
  (GPL) is one example of an open-source license, but in fact there are
  many variants of open-source licensing; you can read about them at
  \url{http://www.opensource.org}. 
\item \textbf{Relational Database} Sometimes abbreviated RDBMS (Relational
  DataBase Management System).  Software that stores a collection of
  data as one or more tables among which
  relationships can be specified.  Microsoft Access is an example of a
  non-open-source RDBMS; MySQL is an example of an open-source RDBMS.
\item \textbf{TCP/IP} Transmission Control Protocol/Internet Protocol.  A
  foundational set of basic Internet communication protocols that has
  dominated information technology for the last two decades.  ``Higher
  level'' protocols such as HTTP are built on top of TCP/IP.  TCP/IP and
  other protocols related to low-level Internet communication 
  are stewarded by the
  Internet Engineering Task Force (IETF), \url{http://www.ietf.org}.
\item \textbf{Tree} A data structure in which each element can have
  zero or more children, but each child has exactly one parent.  Trees
  are commonly used to represent hierarchical data organization; for
  example, many corporate organizational charts are trees, with the
  president at the top, one or more vice-presidents under the president,
  one or more department heads under each vice-president, etc.  XML is
  particularly well suited to representing tree-structured data.
\item \textbf{URI} Universal Resource Identifier, a specially-formatted string
  that names a resource available on the World Wide Web.  One type of
  URI is a Universal Resource Locator (URL) such as
  \texttt{http://www.mtconnect.org}. 
\item \textbf{Web browser} A program running on your PC that allows you to
  contact a Web server via the Internet and view
  the content provided by Web servers.  Microsoft Internet Explorer and
  Firefox are examples of Web browsers.
\item \textbf{Web server} A computer connected to the public Internet that runs
  software that serves Web content such as pages, images, etc. to a Web
  browser.  Apache and Microsoft Internet Information Server (IIS) are
  examples of Web servers.
\item \textbf{XML} eXtensible Markup Language, a human-readable, ASCII-based way
  to represent hierarchically-structured data.  An XML~\emph{document}
  is a self-contained, standalone ``blob'' of XML, such as the
  description of the structure of a piece of equipment or a finite collection
  of measurement samples.
\item \textbf{XML element} A logical ``chunk'' of content within an XML
  document. Elements can be \emph{nested}, that is, an element can
  contain other elements.
\item \textbf{XML tag} In an XML document, a string of text
  surrounded by angle brackets ($<$, $>$) that indicates
  structure (as opposed to content), typically the boundaries of an XML
  element.  For example, the tag
  \verb+<Devices>+ indicates the beginning of the Devices element in an
  XML document, while \verb+</Devices>+ (note the forward slash)
  indicates the end of the Devices
  element; hence the former is usually called an \emph{begin tag} and the
  latter an \emph{end tag} for Devices.  Everything between the begin
  tag and end tag for an element is the content of that element, which
  may in turn contain other elements set off by begin/end tags, and so on.
\item \textbf{XPath} A standard that codifies the syntax for specifying how to
  navigate and retrieve elements from an XML document.
\item \textbf{XML Schema} A standard for describing the a family of XML
  documents in terms of the  specific types of data
  they can include, how those data are represented, and the hierarchical
  structure of  the data.  The \mtc{} specification for data exchange is
  codified as an XML~Schema document.
\end{itemize}
